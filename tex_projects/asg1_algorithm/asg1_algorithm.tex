\documentclass[11pt]{article}

\usepackage{algorithm}
\usepackage{algpseudocode}

\pagestyle{empty}
\textwidth	165mm
\textheight	248mm
\topmargin	-13mm
\oddsidemargin	-2mm
\evensidemargin	-2mm




\begin{document}

\begin{center}
{\sc The University of Melbourne
\\
School of Computing and Information Systems
\\ 
COMP90038 Algorithms and Complexity}
\bigskip \\
{\Large\bf Assignment 1, Semester 2, 2017}
\bigskip \\
%{\large Released: 9 August.  Deadline: 31 August at 23:00
%\bigskip \\
{Student Name: Xiaowei Xu 

Student No.: 843643}
\end{center}

\section*{My Answers for the Challenges}

\begin{enumerate}
\item

\begin{enumerate}
\item
%1(a)
According to the Master Theorem, with $T(n)=2\cdot T(n/2)+1$, we have $a=2, b=2, f(n)=1$, thus we have $d=0$. With $T(1)=1$, we compare $a$ with $b^d$, then we have $a>b^d$. Thus, $T(n) \in \Theta(n^{log_2 2})=\Theta(n)$.

\item
%1(b)
According to telescoping, we have $T(n)=2\cdot T(n)+1=2\cdot (2\cdot T(n/4)+1)+1=4\cdot (2\cdot T(n/8)+1)+2+1$. Thus, according to smoothness rule, we conclude a similar form of the equation, that is: $T(n)=2^k\cdot T(n/2^k)+2^k-1, (k \in [0, {log_2 n}])$. When $n=2^k$, we have $T(1)=1$, thus we have $T(n)=2\cdot n-1$.
\item
%1(c)
I don't agree with the argument.

There's a hidden assumption in the question that $T(n)=k\cdot n$. Thus, it can have $T(n)=2\cdot k \cdot n/2+1$. We can find that $T(n)=2\cdot k \cdot n/2+1 \leq k \cdot n$ is clearly false. Thus, the hidden assumption $T(n)=k\cdot n$ is false. According to the question, from $T(n)\in \Theta(n)$, we should have $T(n)\leq k\cdot n$ but not $T(n)=k\cdot n$. Therefore, we can't have $T(n)=2\cdot k \cdot n/2+1$ and also $2\cdot k \cdot n/2+1 \leq k\cdot n$. Hence we can't have $T(n)\notin \Theta(n)$.
~
\end{enumerate}
~
\item
%2(a)
\begin{enumerate}
\item
%\begin{algorithm}
%\caption{\sc CountSubstring}\label{CountSubtring1}
\begin{algorithmic}[1]
\Function{CountSubstring}{$S[0..n-1]$}
	\State //Brute-force version
	\State //\textbf{Input:} String $S[0..n-1]$
	\State //\textbf{Output:} The number of the substring \textquotedblleft AC"
	\State $count\gets 0$
	\For{$i\gets 0$ \textbf{to} $n-2$}
		\If{$s[i] =$ \textquotedblleft A"}
			\For{$j\gets i+1$ \textbf{to} $n-1$}
				\If{$s[j] =$ \textquotedblleft C"}
					\State $count\gets sum+1$
				\EndIf
			\EndFor
		\EndIf
	\EndFor
	\State \textbf{return} $count$
\EndFunction
\end{algorithmic}
%\end{algorithm}
~

%\pagebreak

%2(b)
\item %See Algorithm \ref{CountSubstring2}

%\begin{algorithm}
%\caption{\sc CountSubstring}\label{CountSubstring2}
%\hrule
%\textbf{Algo1}\space\textsc{FatPig}
%\hrule
\begin{algorithmic}[1]
\Function{CountSubstring}{$S[0..n-1]$}
	\State //Linear version
	\State //\textbf{Input:} String $S[0..n-1]$
	\State //\textbf{Output:} The number of the substring \textquotedblleft AC"
	\State $countA\gets 0$
	\State $countAC\gets 0$
	\For{$i\gets 0$ \textbf{to} $n-1$}
		\If{$s[i] =$ \textquotedblleft A"}
			\State $countA\gets countA+1$
		\EndIf
		\If{$s[i] =$ \textquotedblleft C"}
			\State $countAC\gets countAC+countA$
		\EndIf
	\EndFor
	\State \textbf{return} $countAC$
\EndFunction
\end{algorithmic}
%\end{algorithm}
\textbf{Complexity Analysis:} In the worst case, the basic operation in this algorithm is the comparisons: whether $s[i] =$ \textquotedblleft A" and $s[i] =$ \textquotedblleft C" or not. The comparisons will conduct once each time when i increment 1 in the for loop. The for loop only iterate once from 0 to n-1. Thus, $C(n)\in O(n), n = |s|$ which is the size of the string, the complexity of the algorithm is linear in $|s|$.

\end{enumerate}
~
%3(a)
\item
\begin{enumerate}
\item
%\begin{algorithm}
%\caption{\sc Distance}\label{Distance}
\begin{algorithmic}[1]
\Function{Distance}{$\langle V, E\rangle, v$}
	\State //Use adjacency list in this algorithm
	\State //\textbf{Input:} Graph $G = \langle V, E\rangle$, node $v$
	\State //\textbf{Output:} An array $dist$ recoding all the other nodes' distance from v
	\State creat an array $dist$ of length $n$, index the array with all the nodes' indexes
	\For{each nodes in }
		\State $dist[i]\gets \infty$
		\State $dist[v]\gets 0$
	\EndFor
	\State \textbf{inject}(\textit{queue}, $v$)
	\While{\textit{queue} \textbf{is not empty}}
		\State $a\gets$ \textbf{eject}(\textit{queue})
		\For{each \textit{edge}$(a, b)$}
			\If{$dist[b] = \infty$}
				\State $dist[b]\gets dist[a]+1$
				\State \textbf{inject}(\textit{queue}, $b$)
			\EndIf
		\EndFor 
	\EndWhile
	\State \textbf{return} $dist$
\EndFunction
\end{algorithmic}
%\end{algorithm}
\textbf{Complexity Analysis:} To find the distance of all other nodes to v, one have to traverse all the nodes and edges in the graph. Considering we are using an adjacency list, which contain only an array of nodes and each node's edge to their neighbour, the size of the adjacency list should be $|V| + |E|$. Considering the iterator will revisit the nodes that it has been visited for at most once in this algorithm, the complexity of this algorithm should be $C(n)\in \Theta(|V|+2|E|)$, which runs linear in the size of the graph.
\bigskip \\
%3(b)
\item
%\begin{algorithm}
%\caption{\sc Hub}\label{Hub}
\begin{algorithmic}[1]
\Function{Hub}{$\langle V, E\rangle$}
	\State //Use adjacency list.
	\State //\textbf{Input:} Graph $G = \langle V, E\rangle$
	\State //\textbf{Output:} An array $hub$
	\State create an array $hub$
	\State $minRemoteness\gets \infty$
	\For{each vertex $v$ \textbf{in} $V$}
		\State $dist\gets$ \textsc{Distance}($\langle V, E\rangle, v$)\Comment{Call function \textsc{Distance} and store the returned value to array $dist$}
		\State $maxDistance\gets 0$
		\For{$j\gets 1$ \textbf{to} $n$}
			\If{$dist[j]\geq maxDistance$}
				\State $maxDistance\gets dist[j]$\Comment{Locate the max distance}
			\EndIf
		\EndFor
		\If{$maxDistance\leq minRemotness$}
			\State $minRemoteness\gets maxDistance$\Comment{Locate the minimal remoteness}
			\State $hub\gets v$
		\EndIf
	\EndFor
	\State \textbf{return} $hub$
\EndFunction
\end{algorithmic}
%\end{algorithm}
\textbf{Complexity Analysis:} Every time we call this function, it calls function \textsc{Distance}, which runs in linear, that is $\Theta(n)$ times. The basic operation of the algorithm is the comparison: whether $dist[j]\geq maxDistance$ or not, which conduct n times, plus one comparison between $maxDistance ≤ minRemotness$, altogether $n+1$ times. These comparisons conduct $n$ times, in which n is the number of nodes. Therefore, in the worst case the complexity of this algorithm is $C(n)\in n(n-1)=O(n^2)$.
\end{enumerate}
~
%4(a)
\item
\begin{enumerate}
\item
%\begin{algorithm}
%\caption{\sc Attractor}\label{Attractor}
\begin{algorithmic}[1]
\Function{FindAttractor}{$A[\cdot,\cdot], n$}
	\State //Use adjacency matrix to present the Graph in the matrix, 
	\State //$A[k, j] = 1$ iff $(k, j) \in E$, $A[k, j] = 0$ iff $(k, j) \notin E$
	\State //\textbf{Input:} matrix $A[\cdot,\cdot]$, each node is labelled 1 to n
	\State //\textbf{Output:} $i$ if $i$ is an \textit{attracter}, $-1$ if the Graph has no attractor
	\For{$k\gets 1$ \textbf{to} $n$}
		\For{$j\gets 1$ \textbf{to} $n$}
			\If{$k \neq j$}
			\EndIf 
			\If{$A[k, j] = 0$ \textbf{and} $A[j, k] = 1$}\Comment{Check if $\forall j, (k, j) \notin E$ and $(j, k) \in E$}
				\State $i\gets k$
			\Else
				\State $i\gets -1$
				\State \textbf{break}\Comment{Break the inner loop and switch to the next node}
			\EndIf
			\If{$i \neq -1$}
				\State \textbf{return} $i$
			\EndIf
		\EndFor
	\EndFor
	\State \textbf{return} $-1$
\EndFunction
\end{algorithmic}
%\end{algorithm}

\textbf{Complexity Analysis:} There are all together $n$ nodes in the adjacency matrix. Thus, the size of the adjacency matrix is $n^2$. The basic operations are the two comparisons whether $A[k, j] = 0$ and $A[j, k] = 1$ or not. In the worst case, one can't find a sink in a graph. This algorithm will iterate every cell in the adjacency matrix and in each iteration compare twice. Thus, the worst case complexity $C(n)\in O(2n^2)=O(n^2)$.
\bigskip \\
%4(b)
\item
%\begin{algorithm}
%\caption{\sc FindAttractor}\label{FindAttractor}
\begin{algorithmic}[1]
\Function{FindAttractor}{$A[\cdot,\cdot], n$}
	\State //Use adjacency matrix to present the Graph in the matrix, 
	\State //$A[k, j] = 1$ iff $(k, j) \in E$, $A[k, j] = 0$ iff $(k, j) \notin E$
	\State //\textbf{Input:} matrix $A[\cdot,\cdot]$, each node is labelled 1 to n
	\State //\textbf{Output:} $i$ if $i$ is an \textit{attracter}, $-1$ if the Graph has no attracter
	\State $i\gets 1$
	\State $j\gets 1$
	\While{$i\leq n$ \textbf{and} $j\leq n$}
		\If{A[i, j] = 1}\Comment{Check if $\exists j \in V$ with $(i,j) \in E$}
			\State $i\gets i+1$\Comment{If true move to the next row}
		\Else
			\State $j\gets j+1$\Comment{Else keep on iterating in the sam row}
		\EndIf
	\EndWhile
	\State \textbf{return} $i$
	\If{\textsc{IsAttractor}($A[\cdot,\cdot], i$) = \emph{true}}\Comment{Call function \textsc{IsAttractor} to check if the returned node is an \emph{attractor}}
		\State \textbf{return} $i$
	\Else
		\State \textbf{return} $-1$
	\EndIf
	\State \textbf{return} $-1$
\EndFunction
\end{algorithmic}
%\end{algorithm}
~
%\begin{algorithm}
%\caption{\sc IsAttractor}\label{IsAttractor}
\begin{algorithmic}[1]
\Function{IsAttractor}{$A[\cdot,\cdot], i$}
	\State //This function is used to check if a given node $i$ is an attractor
	\State //\textbf{Input:} The adjacency matrix A, a given node $i$
	\State //\textbf{Output:} \emph{true} if the node $i$ is an \emph{attractor}, \emph{false} if it isn't an \emph{attractor}
	\For{$k\gets 1$ \textbf{to} $n$}
		\If{$A[i, k]\neq 0$}\Comment{Check if $\forall k, (i, k) \notin E$}
			\State \textbf{return} \emph{false}
		\EndIf
	\EndFor
	\For{$j\gets 1$ \textbf{to} $n$}
		\If{$j\neq i$}
			\If{$A[j, i]\neq 1$}\Comment{Check if $\forall j, (j, i) \in E$}
				\State \textbf{return} \emph{false}
			\EndIf
		\EndIf
	\EndFor
		\State \textbf{return} \emph{true}
\EndFunction
\end{algorithmic}
%\end{algorithm}
\textbf{Complexity Analysis:} There are all together $n$ nodes in the adjacency matrix. The complexity of function \textsc{IsAttractor} is $c(n)\in \Theta(2n-1)$, because we only iterate one row and one column(except the node itself when iterating the column) of the matrix with the size n. The worst case complexity of the function \textsc{FindAttractor} is also $c(n)\in O(2n)$, because in the worst case, the iterator may have to skip every row until the last row and iterate all items in the last row, which is $2n$ times of iterations. In the worst case($i=|V|$), everytime we call the function \textsc{FindAttractor}, we must also call the function \textsc{IsAttractor}. Hence, the worst case complexity for function \textsc{IsAttractor} is $C(n)\in O(4n-1)=O(n)$, which runs in linear time.
\end{enumerate}

\end{enumerate}
\end{document}
