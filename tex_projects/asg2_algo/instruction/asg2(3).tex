\documentclass[11pt]{article}

\usepackage{pstricks,pst-node,pst-tree}
\usepackage[noend]{algpseudocode}

\pagestyle{empty}
\textwidth	165mm
\textheight	252mm
\topmargin	-13mm
\oddsidemargin	-2mm
\evensidemargin	-2mm
\renewcommand{\baselinestretch}{1.03}

\newcommand{\id}[1]{\mbox{\textit{#1}}}
\newcommand{\tuple}[1]{\langle #1 \rangle}

\begin{document}

\begin{center}
{\sc The University of Melbourne
\\
School of Computing and Information Systems
\\
COMP90038 Algorithms and Complexity}
\bigskip \\
{\Large\bf Assignment 2, Semester 2, 2017}
\bigskip \\
{\large Released: 23 September.  Deadline: 13 October at 23:00}
\end{center}

\section*{Objectives}

To improve your understanding of data structures and algorithms for
sorting and search.
To consolidate your knowledge of trees and tree-based algorithms.
To develop problem-solving and design skills.
To develop skills in analysis and formal reasoning about
complex concepts.
To improve written communication skills; in particular the
ability to use pseudo-code and present algorithms clearly, 
precisely and unambiguously.

\section*{Five Challenges}

\subsection*{Challenge 1 \hfill {\small (1 mark)}}

Draw a valid AVL tree of height 5, consisting of the 20 integers
from 1 to 20 (no repeats).
Note that a tree of height 5 has some root-to-leaf path with
\emph{six} nodes along it.

\subsection*{Challenge 2 \hfill {\small (2 marks)}}

Snarkbusters Inc.\ produces and deploys snark detectors, 
to be used on building facades.
These are very expensive devices.
One of the company's workers dropped a snark detector while working on the 
16th floor of a building in Melbourne's CBD, and the device broke when it
hit the ground, in spite of having a reputation as indestructible.
The company now wants to know how tough their snark detector really is.
They have hired a consultant, Dr.\ Eva Luator, to determine the 
largest floor level from which a snark detector can be dropped
without breaking.  
That is, they want the number $n$ such that dropping a snark detector
from level $n$ is safe (it does not break), but dropping it from
level $n+1$ is unsafe (it will break).
They call that number $n$ the ``safe limit''.
They know that there is no problem when the device is dropped from ground
level, that is, they know that $0 < n < 16$.

Since there are 15 floor levels to test, Dr.\ Luator asks for a 
handful of snark detectors to use in the experiment.
Once a snark detector is broken it cannot be used again in the test.
But the company will not give her that many. 
They argue that one is enough, because Dr.\ Luator can just drop it from
level 1, and then, if it did not break, drop it from level 2, and so on.
That process will identify the safe limit.
Dr.\ Luator protests: that could make the testing too time consuming 
(and expensive) because, in the worst case, she would need to perform
15 experiments (drops).
As a compromise, it is decided that Dr.\ Luator can have two snark 
detectors to use in her testing.

Eva Luator wants to make the most of the two snark detectors.
She wants to minimise the number $d$ of experiments (drops) that she
needs to do, \emph{in the worst case}, to determine the safe limit.
What is that number $d$, and what is the testing strategy that achieves it?

\pagebreak
\subsection*{Challenge 3 \hfill {\small (3 marks)}}

Consider the following problem.
We are given two arrays, $A[1]..A[n]$ and $B[1]..B[n]$, 
each holding positive integer keys (possibly with duplicates).
The arrays are of the same size, and each is sorted.
We are also given a positive integer $s$, and we want to find a
pair $(i,j)$ of indices such that $A[i] + B[j] = s$.
If several such pairs exist, we do not care which is produced---any
such pair will do.
If no such pair exists then we want that indicated, 
in the form of $(0,0)$, a non-valid pair of indices, being returned.
Using neat, clean pseudo-code, write three different functions to 
solve the problem:
\begin{enumerate}
\setlength{\itemsep}{-0.4ex}
\item
A brute-force function $\textsc{F1}(A[\cdot],B[\cdot],n,s)$.
The worst-case running time of \textsc{F1} should be $\Theta(n^2)$.
Make \textsc{F1} as simple and neat as you can.
\item
A function
$\textsc{F2}(A[\cdot],B[\cdot],n,s)$.
The worst-case running time of \textsc{F2} should be $\Theta(n \log n)$.
Make \textsc{F2} as simple and neat as you can.
\item
A function $\textsc{F3}(A[\cdot],B[\cdot],n,s)$.
The worst-case running time of \textsc{F3} should be $\Theta(n)$.
Make \textsc{F3} as simple and neat as you can.
\end{enumerate}

\subsection*{Challenge 4 \hfill {\small (2 marks)}}

A $d$-dimensional vector $\tuple{u_1,u_2,\ldots,u_d}$ \emph{screens}
the vector $\tuple{v_1,v_2,\ldots,v_d}$ iff there is a permutation
$\pi$ on $\{1,2,\ldots,d\}$ such that
$u_1 > v_{\pi(1)}$,
$u_2 > v_{\pi(2)}$,
\ldots,
$u_d > v_{\pi(d)}$.
For example, $\tuple{7,9,4}$ screens $\tuple{7,6,3}$ 
because one permutation of the latter is $\tuple{6,7,3}$.
On the other hand, $\tuple{7,9,4,5}$ does not screen $\tuple{7,6,3,5}$.
The last example also shows that we can have vectors 
$\mathbf{u}$ and $\mathbf{v}$, neither of which screens the other.

Using pseudo-code, 
give an algorithm which determines, given two vectors $\mathbf{u}$
and $\mathbf{v}$, whether $\mathbf{u}$ screens $\mathbf{v}$.
It should return True iff $\mathbf{u}$ screens $\mathbf{v}$.
If it helps, you an assume that vectors are arrays.
Make the algorithm as efficient as you can, and state,
as precisely as you can, its worst-case time complexity
(and, if possible, average-case as well).

\subsection*{Challenge 5 \hfill {\small (2 marks)}}

An array-based implementation of priority queues is not ideal if we
often need to ``merge'' several priority queues into one, or if we
are programming in a language that does not have arrays.
Here we are interested in a data structure that is \emph{adaptive}
and supports efficient merging, by which we mean taking two
priority queues $P$ and $P'$ and forming a new priority queue with 
elements $P \cup P'$.
As is often the case with adaptive data structures, some operations
may be worst-case expensive, but when we average their cost across many
applications, they are cheap.
That will be the case for the operations we want to implement here:
\textsc{Merge}, \textsc{Insert}, and \textsc{DeleteMax};
all can be made to have $O(\log n)$ worst-case behaviour, amortised.

Your task is to reconstruct pseudo-code for the three operations
(including some helper functions).
The appendix contains a scrambled solution,
with lines of pseudo-code in random order, without any indentation.
You are asked to reconstruct the original code, with proper indentation.

The data structure we use is a kind of \emph{multiway tree},
that is, a node can have any number of subtrees.
Since it is a multiway tree and also a (max-)\emph{heap},
we will call it a \emph{meap}.
A meap $M$ can be empty, in which case we denote it \textbf{void}.
If $M$ is non-empty then $M$ has a key, plus a list of non-empty 
sub-trees (sub-meaps).
Note that the list of subtrees can be of any length; 
in particular it can be empty.

A \emph{tree list} $L$ can be empty, in which case we denote it 
\textbf{null}.
If $L$ is non-empty, then $L$ has a first tree, $L.\id{elt}$ and
a tail, $L.\id{next}$, with the remaining trees.
In our case, these trees are all meaps.

A meap $M$ (other than \textbf{void}) has two attributes:
\begin{itemize}
\setlength{\itemsep}{-0.5ex}
\item $M.\id{key}$ is the root node's key.
\item $M.\id{subtrees}$ is a tree list---the list of $M$'s sub-meaps.
\end{itemize}
A meap must satisfy the \emph{heap condition}:
The key of a child node is smaller than the key of its parent.
That is, for every meap $M'$ in $M.\id{subtrees}$,
we have $M'.\id{key} \leq M.\id{key}$.
The heap condition must hold recursively, for all nodes.

Below are two meaps, $M_1$ (on the left) and $M_2$ (on the right):
\begin{center}
\pstree[nodesep=2pt,levelsep=25pt,treesep=4mm]{\TR{92}}{
  \pstree[treesep=2mm]{\TR{75}}{
    \TR{16}
    \TR{52}
    \TR{27}
  }
  \TR{33}
  \pstree{\TR{66}}{
    \TR{44}
  }
  \pstree{\TR{49}}{
    \TR{7}
  }
  \TR{10}
}
\qquad \qquad
\pstree[nodesep=2pt,levelsep=25pt,treesep=4mm]{\TR{81}}{
  \TR{30}
  \pstree{\TR{64}}{
    \pstree{\TR{57}}{
      \TR{23}
      \TR{44}
    }
  }
}
\end{center}

\vspace{1ex}\noindent
Here $M_1.\id{key} = 92$ and $M_1.\id{subtrees}$ is a list of five children.
These children are meaps with roots 75, 33, 66, 49, and 10.

To find and return the largest key in a meap is easy---it is just the root's
key; we will not be worried about that operation in this question.

To merge two meaps, we include the meap that has the smaller root key
as the leftmost subtree of the other meap.
Merging the two meaps above, we get $\textsc{Merge}(M_1,M_2)$,
shown here:
\begin{center}
\pstree[nodesep=2pt,levelsep=25pt,treesep=5mm]{\TR{92}}{
  \pstree{\TR{81}}{
    \TR{30}
    \pstree{\TR{64}}{
      \pstree{\TR{57}}{
        \TR{23}
        \TR{44}
      }
    }
  }
  \pstree[treesep=2mm]{\TR{75}}{
    \TR{16}
    \TR{52}
    \TR{27}
  }
  \TR{33}
  \pstree{\TR{66}}{
    \TR{44}
  }
  \pstree{\TR{49}}{
    \TR{7}
  }
  \TR{10}
}
\end{center}

\vspace{1ex}\noindent
Once \textsc{Merge} has been implemented, \textsc{Insert} is easy.
\textsc{DeleteMax} removes the largest element from a meap.
It also makes use of \textsc{Merge} but is more complicated.
After discarding the root it must merge the children, to return a
single meap.
We do the merging in two passes.
First we merge the children in pairs from left to right
(that is, the first child with the second, the third with the fourth, 
and so on).
Then we merge the resulting meaps in a single sweep from right to left.

Continuing our example, removal of the root (92) leaves us with 
six meaps to merge:
\begin{center}
  \pstree[nodesep=2pt,levelsep=25pt,treesep=5mm]{\TR{81}}{
    \TR{30}
    \pstree{\TR{64}}{
      \pstree{\TR{57}}{
        \TR{23}
        \TR{44}
      }
    }
  }
  \quad
  \pstree[nodesep=2pt,levelsep=25pt,treesep=5mm]{\TR{75}}{
    \TR{16}
    \TR{52}
    \TR{27}
  }
  \quad
  \pstree[nodesep=2pt,levelsep=25pt,treesep=5mm]{\TR{33}}{}
  \qquad
  \pstree[nodesep=2pt,levelsep=25pt,treesep=5mm]{\TR{66}}{
    \TR{44}
  }
  \qquad
  \pstree[nodesep=2pt,levelsep=25pt,treesep=5mm]{\TR{49}}{
    \TR{7}
  }
  \qquad
  \pstree[nodesep=2pt,levelsep=25pt,treesep=5mm]{\TR{10}}{}
\end{center}

\vspace{1ex}\noindent
Merging pairwise, from left to right, we get:
\begin{center}
  \pstree[nodesep=2pt,levelsep=25pt,treesep=5mm]{\TR{81}}{
    \pstree[treesep=2mm]{\TR{75}}{
      \TR{16}
      \TR{52}
      \TR{27}
    }
    \TR{30}
    \pstree{\TR{64}}{
      \pstree{\TR{57}}{
        \TR{23}
        \TR{44}
      }
    }
  }
  \quad
  \pstree[nodesep=2pt,levelsep=25pt,treesep=5mm]{\TR{66}}{
    \TR{33}
    \TR{44}
  }
  \qquad
  \pstree[nodesep=2pt,levelsep=25pt,treesep=5mm]{\TR{49}}{
    \TR{10}
    \TR{7}
  }
\end{center}

\pagebreak
\noindent
Now, merging the resulting meaps in a single pass from right to left,
we get:
\begin{center}
  \pstree[nodesep=2pt,levelsep=25pt,treesep=5mm]{\TR{81}}{
    \pstree[treesep=2mm]{\TR{66}}{
      \pstree{\TR{49}}{
        \TR{10}
        \TR{7}
      }
      \TR{33}
      \TR{44}
    }
    \pstree[treesep=2mm]{\TR{75}}{
      \TR{16}
      \TR{52}
      \TR{27}
    }
    \TR{30}
    \pstree{\TR{64}}{
      \pstree{\TR{57}}{
        \TR{23}
        \TR{44}
      }
    }
  }
\end{center}

\vspace{1ex}\noindent
Now have fun re-assembling the code from the appendix.
\hfill

\section*{Submission and evaluation}
All problems should be solved and submitted by students individually.
The deadline is 13~October at 23:00.
Late submission will be possible, but a late submission penalty will
apply: a flagfall of 2 marks, and then 1 mark per 12 hours late.

Submit a PDF document via the LMS.
This document should be no more than 2 MB in size.
As usual, for those who want to use the assignment to practise with 
\LaTeX, I will leave the \LaTeX\ source for this document in the
content area where you find the PDF version.
If you produce an MS Word document, it must be exported
and submitted as PDF, and satisfy the space limit of 2 MB.
We also accept \emph{neat} hand-written submissions, but these must be
scanned and provided as PDF, and the scanner resolution must be set low
enough to meet the 2 MB size limit.
Please do not scan/submit your document in landscape mode, or rotated
90 or 180 degrees.
We do not print the submissions; we mark them online.

Make sure you have enough time towards the end of the assignment
to present your solutions carefully.
A good presentation is often more time consuming than solving
the problems.
Explain your algorithms and write neat pseudo-code.
Cormen \emph{et al.}\ (available as an e-book in the library)
provide some guidelines for pseudo-code (pages 20--22).
Use well-chosen function and variable names, and use a consistent layout.
Take care to ensure that the scope of loops and conditionals is
indicated clearly through proper indentation, or the use of 
curly braces, or similar.
If a variable is global, state that.

Your solutions will count towards 10 of the 30 marks allocated outside the
written exam.
We expect your work to be neat---parts of your submission
that are difficult to read or decipher will be deemed incorrect.
Marks are primarily allocated for correctness, but elegance of
algorithms and how clearly you communicate your thinking will 
also be taken into account.
Where indicated, the complexity of algorithms also matters.

Time you put in early will usually turn out to be more productive that
a last-minute effort.
If you get stuck, email Yuan or Harald a precise description of the
problem, bring it up at a lecture, or use the LMS discussion board.
The COMP90038 LMS discussion forum is both useful and appropriate
for this; soliciting help from sources other than the above
will be considered cheating and will lead to disciplinary action.
For more detail about the School's expectations to academic integrity,
click the ``Academic Integrity'' menu item on the subject's LMS site.

\begin{flushright}
Harald S{\o}ndergaard
\\ 23 September 2017
\end{flushright}
\hfill

\pagebreak
\section*{Appendix: Scrambled solution for Question 5}
\begin{algorithmic}
\State \textbf{else}
\State \textsc{Error}(``Cannot delete from empty tree'')
\State \textbf{function} $\textsc{AddToList}(M,L)$ : $\id{TreeList}$
\State \textbf{function} $\textsc{DeleteMax}(M)$ : $\id{MultiwayTree}$
\State \textbf{function} $\textsc{Insert}(x,M)$ : $\id{MultiwayTree}$
\State \textbf{function} $\textsc{Merge}(M_1,M_2)$ : $\id{MultiwayTree}$
\State \textbf{function} $\textsc{MergePairs}(L)$ : $\id{MultiwayTree}$
\State \textbf{if} $L = \mathbf{null}$ \textbf{then}
\State \textbf{if} $L_1 = \mathbf{null}$ \textbf{then}
\State \textbf{if} $M = \mathbf{void}$ \textbf{then}
\State \textbf{if} $M_1 = \mathbf{void}$ \textbf{then}
\State \textbf{if} $M_1.\id{key} > M_2.\id{key}$ \textbf{then}
\State \textbf{if} $M_2 = \mathbf{void}$ \textbf{then}
\State $L_1 \gets L.\id{next}$
\State $L_2 \gets L_1.\id{next}$
\State $M \gets \mathbf{new}\ \id{MultiwayTree}$
\State $M.\id{key} \gets M_1.\id{key}$
\State $M.\id{key} \gets M_2.\id{key}$
\State $M.\id{subtrees} \gets \textsc{AddToList}(M_1,M_2.\id{subtrees})$
\State $M.\id{subtrees} \gets \textsc{AddToList}(M_2,M_1.\id{subtrees})$
\State $M' \gets \mathbf{new}\ \id{MultiwayTree}$
\State $M'.\id{key} \gets x$
\State $M'.\id{subtrees} \gets \mathbf{null}$
\State $M_1 \gets L.\id{elt}$
\State $M_2 \gets L_1.\id{elt}$
\State $R \gets \mathbf{new}\ \id{TreeList}$
\State $R.\id{elt} \gets M$
\State $R.\id{next} \gets L$
\State \textbf{return} $M$
\State \textbf{return} $M_1$
\State \textbf{return} $M_1$
\State \textbf{return} $M_2$
\State \textbf{return} $\textsc{Merge}(M',M)$
\State \textbf{return} 
       $\textsc{Merge}(\textsc{Merge}(M_1,M_2),\textsc{MergePairs}(L_2))$
\State \textbf{return} $\textsc{MergePairs}(M.\id{subtrees})$
\State \textbf{return} $R$
\State \textbf{return} $\mathbf{void}$
\end{algorithmic}

\end{document}
